\documentclass[11pt]{article}
\usepackage[left=3cm, right=3cm, top=2cm]{geometry}
\usepackage{graphicx}
\usepackage[utf8]{inputenc}
\usepackage[english]{babel}
\usepackage{float}

% Prevent heading numbers
\setcounter{secnumdepth}{0}

% Command definitions 
\newcommand{\fitnessplotcaption}[2]{\caption{Global Best Fitness versus Fitness Evaluations for the \textbf{{#1}}, Randomly Generated World. The figure was generated with data obtained by running the GP with the \textbf{{#2}} configuration file.}}

\newcommand{\addgraphic}[1]{\centerline{\includegraphics[scale=0.7]{output/#1_graph.png}}}

\newcommand{\tablecaption}[1]{\caption{Statistical Analysis performed on the {#1}}}

\title{COMP SCI 5401 FS2018 Assignment 2b}
\author{William  Lorey \\ wwlytc@mst.edu}
\date{}

\begin{document}

\maketitle

\tableofcontents

\section{Main Assignment Deliverables}

\begin{figure}[H]
    \addgraphic{small}
    \fitnessplotcaption{Small}{small.cfg}
    \label{fig:small}
\end{figure}

\begin{figure}[H]
    \addgraphic{large}
    \fitnessplotcaption{Large}{large.cfg}
    \label{fig:large}
\end{figure}


\section{BONUS1: Multiple Pacmen Employing the Same Controller}

Support for multiple pacmen was included the main assignment, however, all standard project 
deliverables were supplied with only one pacman spawned. For BONUS1, two pacmen were
spawned for each world, each employing the same controller. Both pacmen had to die
for each game to end and both pacmen shared the fitness (or score) obtained after
completing a game. To give the pacmen an opportunity for a sense of space relative to each other, 
an additional terminal was 
added to the GP tree's terminal set, namely the Manhattan distance between the 
current pacman and the nearest other pacman.

Changes from the main assignment to this bonus assignment involved a mixture of 
cosmetic changes, such as renaming a function \texttt{move\_pacman} to \texttt{move\_pacmen}
to better describe the situation, and real algorithmic changes. One such algorithmic
change was the game over check. As per the bonus specification, each game would
only end if all pacmen were dead, or if the other pre-existing conditions were met. 
To satisfy this, As soon as a pacman was seen as
either colliding with a ghost or sharing a cell with a ghost, it was removed from
that world's list of pacmen. Only after the length of that list was zero would the game
end, not including all pills being consumed and the time running out.

Another algorithmic change involved adding an additional Manhattan distance terminal node
for the distance to the nearest pacman. This was implemented by mirroring the implementation of existing
distance terminal nodes, implementing another distance calculation. Since the main assignment 
encoded a given world's pacman as a member of a list,
adding additional pacmen was not particularly intensive.

Figure \ref{fig:bonus_small} depicts the evaluations versus fitness for the experiments involving
two pacmen per world employing the same controller. This figure offered fairly small variance with
outliers demonstrated as performing far beyond the average of the population in some cases.

Note that only the small world was used to examine BONUS1.

\begin{figure}[H]
    \addgraphic{BONUS_small}
    \fitnessplotcaption{BONUS1 (Multiple Pacmen Employing the Same Controller) Small}{BONUS\_small.cfg}
    \label{fig:bonus_small}
\end{figure}


\section{BONUS2: Multiple Pacmen Employing Different Controllers}

The second bonus involved implementing support for multiple pacmen in a world as well
as the ability for each pacman to be controlled by different controllers. This functionality
in addition to the new game over check and terminal set addition discussed in the BONUS1 
section was implemented.

The implementation of this bonus section required more changes to the original codebase than BONUS1,
as the ties between the pacman list associated with each world and the single pacman
controller were stronger. The main way this dependency was removed was by creating 
a list of pacman controllers to match the list of pacmen. A pacman list index was then passed along 
with the game's state to the pacman controllers so that controller had a specific pacman
to control. Code also had to be removed that allowed one controller to move all pacmen
passed along in the game state, as each pacman now had its own controller.

Because there were multiple controllers, functions such as the bloat controller had
to be adjusted to handle multiple controller trees. Additionally, the recombination and mutation functions
were updated so that each controller tree was involved in crossover and sub-tree mutation. 

To maintain the relationship between pacman and controller, dead pacmen, instead of being
removed from the pacman list, were noted in a taboo list. Before moving pacmen, this list
would be polled to see if the current pacman being moved was dead. Once the length of the
taboo list equaled the length of the pacman list, all pacmen in the world were known to be 
dead.

One additional change was made to satisfy this bonus section. The solution file generator
was altered to output the controller of both pacmen upon the discovery of a globally optimal
score.

Figure \ref{fig:bonus_small_multi}, depicting the evaluations versus fitness for this bonus
assignment with two independently-controlled pacmen per world, demonstrates volatile 
variances and high outlying score values.

Note that only the small world was used to examine BONUS2.


\begin{figure}[H]
    \addgraphic{BONUS_small_multi_controller}
    \fitnessplotcaption{BONUS2 (Multiple Pacmen Employing Different Controllers) Small}{BONUS\_small\_multi\_controller.cfg}
    \label{fig:bonus_small_multi}
\end{figure}


\section{Analysis of Main Assignment, BONUS1, and BONUS2}

Three pairings of statistical analysis were performed on the main assignment GP, BONUS1, and BONUS2. This analysis
consisted of an f-test to determine the assumption of equal variances followed by a t-test either assuming equal variances
or assuming unequal variances depending on the output of the f-test. The pairings of statistical analysis are as follows:

\begin{itemize}
  \item BONUS1 Small, Multi-Pacman World versus BONUS2 Small, Multi-Pacman World Employing Multiple Controllers (Table \ref{bonus1_bonus2})
  \item BONUS2 Small, Multi-Pacman World Employing Multiple Controllers versus Small, Single-Pacman World (Table \ref{bonus2_std})
  \item BONUS1 Small, Multi-Pacman World versus Small, Single-Pacman World (Table \ref{bonus1_std})
\end{itemize}

In each analysis it was involved in (Tables \ref{bonus1_bonus2} and \ref{bonus2_std}), the BONUS2 approach, 
employing multiple pacmen each with their own controller,
proved to be statistically superior on the given problem space. This implies that the pacmen controllers could have evolved
in tandem to ensure cooperation between pacmen in getting the highest possible score.

Another interesting observation is that when the standard GP employing a single pacman was compared to BONUS1, employing multiple
pacmen and one pacman controller (Table \ref{bonus1_std}), there was no statistical difference between the two approaches.
This result showed that more pacmen, all controlled by the same controller, are not necessarily more optimal than one pacman
controlled by one controller and that to see the gains of more pacmen in the world, alternate controller approaches (such
as using different controllers for each pacman, as seen in BONUS2) may be necessary.


\begin{table}[H] 
\tablecaption{BONUS1 Small, Multi-Pacman World versus BONUS2 Small, Multi-Pacman World Employing Multiple Controllers}        
\label{bonus1_bonus2}                 
\resizebox{\textwidth}{!}{%        
\begin{tabular}{|l|l|l|}           
\hline               
  & BONUS\_small\_multi\_controller & BONUS\_small  \\ \hline 
 mean & 142.96666666666667 & 101.1 \\ \hline 
 variance & 439.96555555555557 & 408.8233333333334 \\ \hline 
 standard deviation & 20.975355910104494 & 20.21938014216394 \\ \hline 
 observations & 30 & 30 \\ \hline 
 df & 29 & 29 \\ \hline 
 F & 1.0761752563590357 &   \\ \hline 
 F critical & 0.5373999648406917 &   \\ \hline 
 Unequal variances assumed &  &    \\ \hline 
  &  &     \\ \hline 
  observations & 30 &  \\ \hline 
  df & 31 &  \\ \hline 
  t Stat & 7.738691619777811 &  \\ \hline 
  P two-tail & 1.694665580406777e-10 &  \\ \hline 
  t Critical two-tail & 2.0395 &  \\ \hline 
  BONUS\_small\_multi\_controller is statistically better than BONUS\_small &  &   \\ \hline 
  \end{tabular}%     
}                    
\end{table}

\begin{table}[H] 
\tablecaption{BONUS2 Small, Multi-Pacman World Employing Multiple Controllers versus Small, Single-Pacman World}        
\label{bonus2_std}                 
\resizebox{\textwidth}{!}{%        
\begin{tabular}{|l|l|l|}           
\hline               
  & BONUS\_small\_multi\_controller & small  \\ \hline 
 mean & 142.96666666666667 & 101.0 \\ \hline 
 variance & 439.96555555555557 & 494.46666666666664 \\ \hline 
 standard deviation & 20.975355910104494 & 22.23660645572221 \\ \hline 
 observations & 30 & 30 \\ \hline 
 df & 29 & 29 \\ \hline 
 F & 0.8897779875061795 &   \\ \hline 
 F critical & 0.5373999648406917 &   \\ \hline 
 Unequal variances assumed &  &    \\ \hline 
  &  &     \\ \hline 
  observations & 30 &  \\ \hline 
  df & 31 &  \\ \hline 
  t Stat & 7.393150913824469 &  \\ \hline 
  P two-tail & 6.520299920112773e-10 &  \\ \hline 
  t Critical two-tail & 2.0395 &  \\ \hline 
  BONUS\_small\_multi\_controller is statistically better than small &  &   \\ \hline 
  \end{tabular}%     
}                    
\end{table}

\begin{table}[H] 
\tablecaption{BONUS1 Small, Multi-Pacman World versus Small, Single-Pacman World}
\label{bonus1_std}                 
\resizebox{\textwidth}{!}{%        
\begin{tabular}{|l|l|l|}           
\hline               
  & BONUS\_small & small  \\ \hline 
 mean & 101.1 & 101.0 \\ \hline 
 variance & 408.8233333333334 & 494.46666666666664 \\ \hline 
 standard deviation & 20.21938014216394 & 22.23660645572221 \\ \hline 
 observations & 30 & 30 \\ \hline 
 df & 29 & 29 \\ \hline 
 F & 0.8267965484697318 &   \\ \hline 
 F critical & 0.5373999648406917 &   \\ \hline 
 Unequal variances assumed &  &    \\ \hline 
  &  &     \\ \hline 
  observations & 30 &  \\ \hline 
  df & 31 &  \\ \hline 
  t Stat & 0.01791782942177858 &  \\ \hline 
  P two-tail & 0.9857664534615012 &  \\ \hline 
  t Critical two-tail & 2.0395 &  \\ \hline 
  Nether small nor & & \\
 BONUS\_small is statistically better &  &   \\ \hline 
  \end{tabular}%     
}                    
\end{table}

\end{document}
